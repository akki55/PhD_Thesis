%!TEX root = ../template.tex
%%%%%%%%%%%%%%%%%%%%%%%%%%%%%%%%%%%%%%%%%%%%%%%%%%%%%%%%%%%%%%%%%%%%
%% annex6.tex
%% NOVA thesis document file
%%
%% 
%%%%%%%%%%%%%%%%%%%%%%%%%%%%%%%%%%%%%%%%%%%%%%%%%%%%%%%%%%%%%%%%%%%%
\chapter{Case study: RDAL and USE-ME integration}
\label{ann:RDAL}

In this Annex we include the article \cite{barisic2017RDAL}, named \textbf{'A Requirements Engineering Approach for Usability-Driven DSL Development'}, which was published in Proceedings of 10th International Conference on Software Language Engineering (SLE) in 2017.

This case study, discussed in Section \ref{sec:RDAL}, was used as a part of \gls{useme} implementation validation of our research process (Figure \ref{fig:research}). It served as illustration of integration of USE-ME approach (Chapter \ref{cha:useme}) with RDAL requirement approach \cite{Blouin2011DefiningSupportMODRE}. This combination of existing languages and tools provides a comprehensive requirement engineering approach for \gls{dsl} development and an interesting case study of languages composition allowing the reuse of the assets of the existing languages. The approach was illustrated with the development of the Visualino (Section \ref{sec:visualino}, Annex \ref{ann:Visualino}).

The project can be found at public GitHub repository\footnote{github.com/akki55/useme (accessed September 19, 2017)}. Submission artefacts%\footnote{https://goo.gl/Syf1jr} 
are published and documented as a part of conference proceeding, and installation instructions are attached after the article.

This work was supported by FCT/MEC NOVA LINCS; PEst UID/ CEC/04516/ 2013 and DSML4MA TUBITAK/0008/2014 Projects, as well as \gls{cost} Action IC1404 \gls{mpm4cps}.

\includepdf[pages={-}]{Chapters/RADAL_UseME_SLE.pdf}
\includepdf[pages={-}]{Chapters/RADALArtefact.pdf}

