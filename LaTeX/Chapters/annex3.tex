%!TEX root = ../template.tex
%%%%%%%%%%%%%%%%%%%%%%%%%%%%%%%%%%%%%%%%%%%%%%%%%%%%%%%%%%%%%%%%%%%%
%% annex3.tex
%% NOVA thesis document file
%%
%% 
%%%%%%%%%%%%%%%%%%%%%%%%%%%%%%%%%%%%%%%%%%%%%%%%%%%%%%%%%%%%%%%%%%%%
\chapter{Case study: FlowSL}
\label{ann:FlowSL}

In this Annex we include the article \cite{barisic2014flows}, named \textbf{'Introducing Usability Concerns  Early  in  the  \gls{dsl}  Development  Cycle  :   FlowSL  Experience  Report'}, which was published in Proceedings of the Model-Driven Development Processes and Practices (MD2P2 ) Workshop at MODELS conference in 2014. 

FlowSL (Section \ref{sec:flowsl}) is a \gls{dsl} for specifying humanitarian campaigns to be conducted by a non-governmental organization \gls{psi} and is integrated into \gls{mvc} platform\footnote{https://movercado.wordpress.com/ (accessed September 19, 2017)}. Work was developed under the collaboration of the Engineering Faculty of Porto (FEUP) and \gls{psi}.
We applied action research to the development of a industrial oriented \gls{dsl} FlowSL,
in order to validate our solution design proposed in Section \ref{sec:iterativeUCD} as a part of our research process (Figure \ref{fig:research}). Final report of this project can be downloaded from public repository \footnote{goo.gl/gQzX7B}.

\includepdf[pages={-}]{Chapters/FlowSL_Models.pdf}
