%%%%%%%%%%%%%%%%%%%%%%%%%%%%%%%%%%%%%%%%%%%%%%%%%%%%%%%%%%%%%%%%%%%%
%% appendix1.tex
%% NOVA thesis document file
%%
%% Chapter with example of appendix with a short dummy text
%%%%%%%%%%%%%%%%%%%%%%%%%%%%%%%%%%%%%%%%%%%%%%%%%%%%%%%%%%%%%%%%%%%%
\chapter{Usability Evaluation Methods}
\label{app:usaMethods}

\begin{itemize}
    \item \textbf{User Testing} - A representative amount of end users interacts with the software following a list of pre-defined tasks. Exhaustive observations of these human-system interactions allow the identification of usability issues related to the system. This evaluation method is commonly applied in a usability lab whose equipment enables the recording of user’s gestures and user’s computer screen for later analysis.
    
    \item \textbf{Heuristic Evaluation} - A group of usability specialists judge whether each dialogue element of the software system follows established usability principles, called “heuristics”.
    
    \item \textbf{Interview} - Both end user and usability specialist participate in a discussion session about the usability of a software application.
    
    \item \textbf{User Testing – Thinking Aloud / Thinking Out Loud} - This version of user testing involves the execution of the “thinking aloud protocol”. Users have to verbalize their thoughts while they interact with the software system. Supervisors should encourage end- users to express their opinions during the activity. In some cases, this indication is only requested at the beginning of the testing.
    
    \item \textbf{Usability Metrics / Software Metrics} - The purpose of this method is to establish quantitative measurements. Usability metrics quantify the usability of a system regarding effectiveness, efficiency and satisfaction. Usually, some equations are used to determine numeric values about the usability of a system. The participation of a representative number of users is required to generalize the obtained results.
    
    \item \textbf{Automated Evaluation via Software Tool} - A software tool is used to perform all the activities that are required in a usability evaluation. Depending on the type of software, this tool can be able to simulate human actions. Other applications only keep track of the user’s activities, and perform metric-based measurements. Additionally, these systems can generate a log file that can be analyzed after the testing.
    
    \item \textbf{Cognitive Walkthrough} - A usability specialist simulates the actions of a novice user of the system. During this interaction, the inspector has to identify potential issues of usability.
    
    \item \textbf{Prototype Evaluation} - Both, an end users and usability specialists participate of a meeting in which users are asked to explain their expectations about a paper prototype or a mockup.
    
    \item \textbf{Focus Group} - A representative group of end users are requested to participate in an open discussion to analyze the graphical interface of a software product. In this method, participants are free to listen and talk to other group members. In this way, they can develop own ideas based on previous comments.
    
    \item \textbf{Checklist Verification} - A usability specialist verifies if a graphical user interface meets a series of well-defined design specifications. A verification checklist helps inspectors to manage all of the details of usability that must be considered in a particular software product. 
    
    \item \textbf{Pencil \& Paper} - The users evaluate aspects of a prototype on paper. They are free to modify the interface design with a pencil. Additionally, they can write their comments and make annotations to specify their observations in detail.
    
    \item \textbf{Perspective Based Usability Inspection} - In each inspection session, the specialist focuses on a specific subset of usability issues covered by one of several usability perspectives. Each perspective provides the inspector a list of questions that represent the usability issues to check and a specific procedure for conducting the inspection. The assumption is that with focused attention and a well-defined procedure, each inspection session can detect a higher percentage of the problems related to the perspective used, and that the combination of different perspectives can uncover more problems than the combination of the same number of inspection sessions using a general inspection technique.
    
    \item \textbf{Field Observation / Field Study} - This method involves a usability specialist observing user’s natural behavior in their “natural habitat”, the field where the daily activity takes place or the workplace where the software product will be implemented. The facilitator gives the user a task and observes, takes notes, and asks questions as the user employs the software product to complete the defined task. Observation can be direct, where the inspector is present during the task, or indirect, using special software to capture user actions on the computer and record the session.
    
    \item \textbf{Eye Tracking} - This method involves measuring either where the user is looking (the point of gaze) or the motion of an eye during the use of a software product. There are several devices to perform this kind of evaluation such as: special monitors, specific cameras, sensors and even specialized software. By analyzing the visual path of the end users across the interface, it is possible to determine the relevant information, the sections that are ignored, the content which is overlooked any other gaze-related question.
    
    \item \textbf{Click Map / Scroll Map / Heat Map} - Clickmaps shows where users click on a software interface. This information allows inspectors to identify the most popular sections, and see which sections users mistake for links. This map is often represented by colors which indicate the amount of clicks in a specific area. A click map can be obtained through the use of a special software tools.
    
    \item \textbf{Opinion Mining} - This method refers to the use of natural language processing and text analysis to identify subjective information regarding the usability of a software product. For this purpose, a representative group of user have to write their opinion for certain software in three usability factors: effectiveness, efficiency and satisfaction. Then, these comments are analyzed using specialized techniques of Computer Science to determine how positive or negative are in each category.
    
    \item \textbf{Web Usability Evaluation Process} - This method involves the decomposition of the usability concept into sub-characteristics and measurable attributes, which are then associated with metrics in order to quantify them numerically. This technique has been specially developed for Web applications. The purpose is to provide feedback during all phases of the software development process. A complete model, including all the sub- characteristics attributes and their associated metrics, is provided by the authors of this method.
    
    \item \textbf{Retrospective Thinking Aloud} - This method is another variant of user testing. It is a similar practice to the thinking aloud protocol, however in this method, users have to verbalize their thoughts after the user testing session activities, instead of during them. Users are requested to use the system and perform certain tasks in silence. Participants verbalize their thoughts afterward while they are watching a recording of their performance.
    
    \item \textbf{Cognitive Task Analysis} - This techniques involves the process of learning about ordinary users by observing their interaction with a specific software product in order to understand in detail how they perform their tasks and achieve their intended goals. Tasks analysis helps to identify the tasks that a software application must support and can also help you refine the navigation or search. This method is focused on understanding tasks that require decision-making, problem-solving, memory, attention and judgment.
    
    \item \textbf{Usability Guidelines} - A group of specialists have to evaluate the graphical interface of a software product according to pre-defined usability guidelines. Although this technique is similar to heuristic evaluation, the procedure is different. In this technique, each inspector can work individually. There is no need to rate the severity and criticality of each usability issue. The assessment tool is not necessarily a set of usability heuristics. Inspectors can even use guidelines that are provided by the software development company.
    
    \item \textbf{Card Sorting} - This method can be used to verify the organization and structure of the information that appears in a software application. For this kind of evaluation, some paper cards are required. Each card has to contain a word or phrase written on one side. This expression has to represent a specific concept that is considered as part of the graphical user interface. Participants are given a stack of cards and are asked to group them together as it makes sense to them. They organize topics into categories and may also help to label these groups. If an accepted and standardized taxonomy becomes visible, it would be appropriate to apply that taxonomy in the interface.
    
    \item \textbf{Canvas Card Sorting} - This technique is a variation of the classical card sorting. This method requires that users select the most valuable concepts and arrange them in a predefined template. In this version, the main categories are previously established, and users only have to place each card into one of the groups.
    
    \item \textbf{Retrospective Sense Making} - This method is based on a retrospective protocol, in which users are asked to verbalize their thoughts after a set of tasks is completed. This specific technique establishes the use of open-ended questions in order to encourage users to process information from the long-term memory, providing justifications and explanations of certain actions they performed during their interaction with an interface. The questions should be oriented to analyze the cognitive process through which people experience problems and choose to perform certain actions, among alternative ones, in order to solve the problems experienced at a specific point in time.
    
    \item \textbf{Personas} - This method involves the description of different fictitious users of the software application. These representations should include a brief profile of goals and characteristics that represent the needs of a larger group of real users. The evaluation involves an analysis of the graphical user interface considering the goals, possible behaviors, attitudes, motivations and business objectives of each profile.
    
    \item \textbf{User Workflow} - This method establishes the elaboration of diagrams to represent all the paths that are available in a software system to perform a specific task. This diagram allows specialists to analyze the achievement of multiple goals which involve many sub- tasks. Additionally, it permits the examination of the different users’ preferences and the order in which certain tasks are performed.
    
    \item \textbf{Cognitive Jogthrough} - This method is an alternative version of the cognitive walkthrough. In this version, while inspectors are working through a series of tasks, they ask themselves a set of questions from the perspective of the user. The answers to these questions should be ranked according to the percentage of potential users are expected to have problems (from 0 to 3 in a Likert scale).
    
    \item \textbf{Domain Specific Inspection} - This method involves the use of a model that can be adapted to any software domain. Specialists should determine the areas and attributes that are more relevant for software they are going to evaluate. The inspection should be performed according to the guidelines that have been established for each usability attribute.
    
    \item \textbf{Participatory Heuristic Evaluation} - is an extension of the traditional heuristic evaluation where some principles are considered to evaluate the graphical user interface. Participatory heuristic evaluation uses the same technique. However, it involves the participation of end users in the evaluation process as “domain expert inspectors”. Additionally, some additional heuristics are added to include some usability aspects that are not considered by the traditional Nielsen’s proposal.
    
    \item \textbf{Semiotic Inspection Method} - The purpose of this method is the analysis of the messages conveyed through the designer-to-user metacommunication. These messages are expressed with a broad range of signs and symbols in the interface, from one or more signification systems. The aim of the semiotic inspection method is the evaluation of these elements, searching for actual or potential problems of communication and redesign opportunities to improve the communication.
    
    \item \textbf{Usability \& Communicability Evaluation Method} - In this method, evaluators have to identify communication breakdowns while a representative amount of users interacts with the product software. There are thirteen expressions of communication breakdown or labels to categorize the problems of communicability and usability. The evaluator should interpret these issues and rebuild the message to identify possible improvements.
    
    \item \textbf{Simplified Pluralistic Walkthrough} - Users and designers participate together in a meeting to evaluate new ideas regarding the graphical user interface of a software product. The method does not require a working prototype. They can develop a design from just ideas. The system designers can get valuable information about the users' tasks in addition to the comments on the design.
    
    \item \textbf{Simplified Streamlined Cognitive Walkthrough} - This method establishes the same procedure than cognitive jogthough. The difference is that evaluators only required asking two questions at each step of the inspection. Moreover, it involved to elaborate less documentation.
    
    \item \textbf{Music Performance Measurement Method} - 	This method establishes that the usability of a product is measured by the extent to which specific users achieve specific goals in a specific environment. Some metrics are employed to determine qualitative data regarding usability. The technique indicates that the controlled experiments should be performed as close to a real work environment. A software tool called DRUM can be used to analyze log files.

\end{itemize}

	
	

	
	
	
	
