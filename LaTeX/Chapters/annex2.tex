%!TEX root = ../template.tex
%%%%%%%%%%%%%%%%%%%%%%%%%%%%%%%%%%%%%%%%%%%%%%%%%%%%%%%%%%%%%%%%%%%%
%% annex2.tex
%% NOVA thesis document file
%%
%% 
%%%%%%%%%%%%%%%%%%%%%%%%%%%%%%%%%%%%%%%%%%%%%%%%%%%%%%%%%%%%%%%%%%%%
\chapter{Case study: RPG DSL}
\label{ann:RPGDSL}

In this Annex we include the article \cite{marques2012rpg} about cases study in which  an \gls{rpg} \gls{dsl} for  product lines was developed, named \textbf{'The \gls{rpg} \gls{dsl}: A Case Study of Language Engineering using \gls{mdd} for Generating \gls{rpg} Games for Mobile Phones'}, which was published in Proceedings of the \gls{dsm} workshop at SPLASH in 2012. 

In this case study, discussed in Section \ref{sec:RPG}, \gls{rpg} \gls{dsl} was completely built using \gls{mdd} software development techniques and served as DSL development example during problem investigation phase of our research process (Figure \ref{fig:research}). 

This work was funded by PEst-OE/EEI/UI0527/2011 Centro de Informatica e Tecnologias da Informacao (CITI/FCT/UNL).

\includepdf[pages={-}]{Chapters/RPG_DSL_DSMworkshop.pdf}
