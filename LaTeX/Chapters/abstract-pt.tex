%!TEX root = ../template.tex
%%%%%%%%%%%%%%%%%%%%%%%%%%%%%%%%%%%%%%%%%%%%%%%%%%%%%%%%%%%%%%%%%%%%
%% abstrac-pt.tex
%% NOVA thesis document file
%%
%% Abstract in Portuguese
%%%%%%%%%%%%%%%%%%%%%%%%%%%%%%%%%%%%%%%%%%%%%%%%%%%%%%%%%%%%%%%%%%%%

A adoção de linguagens específicas de domínio (DSLs) é considerada uma abordagem para reduzir a complexidade acidental do desenvolvimento de sistemas de software. A disponibilidade ferramentas recentes e sofisticadas de suporte ao desenvolvimento de linguagens ("modelling workbenches") tem tornado as DSLs populares. No entanto, esta popularidade tem colocado em evidência os riscos de uma DSL mal projetada. Uma DSL mal concebida pode causar mais danos e reduzir a produtividade, em comparação com alternativa existentes. De facto, uma DSL problematica pode ser muito difícil de adotar pelos seus utilizadores. Como tal, a preocupação com a Usabilidade é uma das principais características para mitigar esse risco, pois tem um impacto significativo na produtividade alcançada dos utilizadores da DSL.

A prática corrente da Engenharia de Linguages ("Software Language Engineering - SLE"), por se focar no desenho e implementação, negligencia a usabilidade de DSLs. Uma questão de investigação relevante em engenharia de Linguagens é de como integrar de forma sistemática no processo de engenharia da linguagem, a preocupação com a usabilidade. Nós argumentamos que, uma abordagem sistemática atempada baseada em técnicas de avaliação experimental da interface com o utilizador, deverá ser usada para avaliar o impacto das DSLs durante seu processo de desenvolvimento. Pretende-se que seja desenrolada enquanto o custo de corrigir os problemas de usabilidade é relativamente baixo em comparação com o que poderá acontecer em fases tardias do processo de desenvolvimento. 

O foco desta dissertação é construir uma abordagem sistemática, que seja integrada com o processo comum de desenvolvimento de DSLs  e que se foque na questão da avaliação de usabilidade envolvendo a todo o momento os utilizadores.

Para ser eficaz, a abordagem sistemática deve basear-se na informação produzida ao longo do processo de engenharia. O desenvolvimento orientado a modelos ( "Model-Driven Development - MDD") permite-nos capturar explicitamente toda a informação relevante ao processo de avaliação usando modelos e estabelecendo links de rastreabilidade entre eles.

Nós propomos o Ambiente de Modelação de Engenharia de Software de Usabilidade (USE-ME) como uma estrutura conceptual para a avaliação de usabilidade de DSLs. Nesta teses definimos, passo-a-passo, o processo de avaliação. Este trabalho demonstra a sua viabilidade tendo-se desenvolvido um protótipo do USE-ME. As instâncias de modelo em USE-ME, fornecem toda a informação necessária de suporte à decisão quanto à usabilidade da DSL, apontando para as oportunidades para a sua melhoria. Finalmente, iremos discutir detalhadamente vários casos de estudo realizados num ambiente empresarial e académico para ilustrar a metodologia proposta.
% Palavras-chave do resumo em Português
%\begin{keywords}
%Palavras-chave (em Português) \ldots
%\end{keywords}
% to add an extra black line
