%!TEX root = ../template.tex
%%%%%%%%%%%%%%%%%%%%%%%%%%%%%%%%%%%%%%%%%%%%%%%%%%%%%%%%%%%%%%%%%%%%
%% annex4.tex
%% NOVA thesis document file
%%
%% 
%%%%%%%%%%%%%%%%%%%%%%%%%%%%%%%%%%%%%%%%%%%%%%%%%%%%%%%%%%%%%%%%%%%%
\chapter{Case study: Visualino}
\label{ann:Visualino}

In this Annex we include the article about evaluation of the \gls{dsl} Visualino (Section \ref{sec:visualino}), for the programming low-cost robots, 
under a title \textbf{'Leveraging teenagers feedback in the development of a Domain-Specific Language -- the case of programming low-cost robots'}. This article is published in \gls{acm} \gls{sac} conference Action IC1404 \gls{mpm4cps} in 2018.
The work was developed under the collaboration between the group ASE NOVA/LINCS and Artica\footnote{http://artica.cc/ (accessed September 19, 2017)}, a company that specialises in the development of robotic and audio-visual solutions. 

The first language design was developed in 2013 in the context of the master thesis \cite{PedroMiguel2013}, where language was named Farrusco. Later, language continue to evolve and was renamed to Visualino after second release in 2015, and recently Gyro, after a third release in 2016. We manage to show that the usability of the Gyro significantly improved in terms of efficiency and satisfaction when compared to earlier version. To consult the details, take a look at experiment repository\footnote{https://sites.google.com/view/vl-empiricalstudy/home (accessed September 19, 2017)}. 

This study present application of our usability evaluation approach (Section \ref{sec:iterativeUCD}) on the industrial case study during several development iteration. 
As this, it served as design validation example of our research process (Figure \ref{fig:research}). Further, it was used to validate feasibility of our approach (Chapter \ref{cha:useme}) as a \gls{useme} model instantiation example (Section \ref{sec:visualinoPrototype}) \footnote{https://github.com/akki55/useme/tree/master/examples/pt.fct.unl.novalincs.useme.example.Visualino}.  
Further, it served as a running example for our integration study (Section \ref{sec:RDAL}, Annex \ref{ann:RDAL}) \footnote{https://github.com/akki55/useme/tree/master/examples/gyro}.

This work was supported by FCT/MEC NOVA LINCS; PEst UID/ CEC/04516/ 2013 and DSML4MA TUBITAK/0008/2014 Projects. 

\includepdf[pages={-}]{Chapters/ACMSAC2018Visualino.pdf}
